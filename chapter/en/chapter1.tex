% chapter1.tex -- en (English)
\chapter{Preface}

Thank you for interesting in {\autor}s {\Bezeichnung}.\\
In addition to a CD player you are also concerned with a complete 
computer based on a {\RPi} which offers you as a hobbyist a lot of other
possibilities. Read this manual to  get to know this device and possibly
even to rebuild it.

\section{Legal Notes}
When designing the {\Bezeichnung}, it was taken care to use only 
software that is provided under a free license such as FreeBSD, GNU GPL 
or similar.

\subsection*{Trademarks}
Some names used in this document may be trademarks. The use of these 
trademarks by third parties for their purposes may infringe the rights 
of the holders.

\subsection*{Links}
This manual contains links to external sites on the internet. Despite
linking the author {\autor} does appropriate these contents, since they 
are not within his sphere of influence! At the time of linking, there 
were no illegal contents noticeable. It is not reasonable for the
author to check the links permanently for any changes that might violate
the law. However, if current or future content should be illegal, the 
author may be contacted by e-mail to \url{mailto:schlizbaeda@gmx.de}.
Appropriate actions will then be taken to remove the affected link(s).

\subsection*{DRM}
The official reading of the inventors of this nuisance is 
\textit{Digital Rights Management}, but groups like the Free Software 
Movement interpret these technical methods as \textit{Digital 
Restriction Management}, since many of these procedures are not 
compatible to the most open source licenses.\\
However, this \textbf{doesn't mean} that DRM can be ignored! Around 2003
in Germany the end consumer(!) was made liable by law to recognize and
obey technically non-functioning DRM (\eg a non-working ``copy 
protection'' on audio CDs not according to the Red Book specification, 
so-called un-CDs) and to take appropriate measures to comply with DRM.\\
In the other EU countries there do exist similar laws.\\
\textbf{On {\Bezeichnung} it cannot be guaranteed that all DRM measures 
will be technically detected and recognized. Therefore they cannot be 
taken into account automatically!} This applies in particular to un-CDs 
labeled with a copy protection icon: Even the private digital copy of 
such a product is strictly speaking illegal but currently these 
``violations'' usually aren't prosecuted by law\dots

\subsection*{FreeBSD License of the audio player {\audacious}\\{\audaciousStable}}
Copyright 2001-2017 Audacious developers and others

Redistribution and use in source and binary forms, with or without modification,
are permitted provided that the following conditions are met:

1. Redistributions of source code must retain the above copyright notice,
   this list of conditions, and the following disclaimer.

2. Redistributions in binary form must reproduce the above copyright notice,
   this list of conditions, and the following disclaimer in the documentation
   provided with the distribution.

This software is provided "{as} is"{ and} without any warranty, express or 
implied. In no event shall the authors be liable for any damages arising from 
the use of this software.

\url{https://audacious-media-player.org/download}

Based on this license, the source code of {\audacious} on  
{\audaciousStable} has been updated for use on the {\RPi} by {\autor}.
That program has got an ``eject'' functionality, since pressing the 
eject hardware button of the {\CDROM} drive sometimes causes hang-ups. 
Further information can be found in chapter \ref{sect:compile}.\\
The source code of the changed software can be downloaded from this
link:\\
\url{https://github.com/schlizbaeda/audacious-raspiblaster}

\subsection*{Picture Licenses}
All relevant photographs and technical illustrations in this document 
originate from the author {\autor} and are hereby published by him under
the \textit{Creative Commons} License \textbf{CC-BY-SA 3.0}. They may 
therefore be used and re-published by anyone -- modified or unmodified 
-- by naming the author under the same conditions:\\
\includegraphics[height=35px]{CC-BY-SA.png}

However this manual contains some icon graphics from other sources using
these licenses:
\begin{table}[!h]
\centering
\renewcommand{\arraystretch}{2}
\begin{tabular}{|l|p{13.0cm}|}
\hline
                                                                    & The author \textit{Larry Ewing} released this picture under the following condition:\\
\parbox[c]{1.2cm}{~\newline \includegraphics[height=35px]{tux.png}} & ``Permission to use and/or modify this image is granted provided you\newline
                                                                        acknowledge me -- lewing@isc.tamu.edu -- and the GIMP if someone asks.''\\
\hline
\parbox[c]{1.2cm}{~\newline \includegraphics[height=35px]{rpiIcon.png}} & Logo of the Raspberry Pi Foundation:\newline \smaller{\url{https://static.raspberrypi.org/files/Raspberry_Pi_Visual_Guidelines_2018.pdf}}\\
\hline
\parbox[c]{1.2cm}{~\newline \includegraphics[height=35px]{audacious.png}} & Desktop icon of {\audacious} audio player:\newline probably CC-BY-SA?\\
\hline
\parbox[c]{1.2cm}{~\newline \includegraphics[height=35px]{raspiblaster.png}} & Alternative desktop icon for an audio player on {\Bezeichnung}:\newline CC-BY-SA U.S. (author: Wallpaper FX, \url{http://www.wallpaperfx.com/})\\
\hline
\end{tabular}
\vspace{0.5cm}
\caption{Licensing of third party graphics}
\end{table}


\section{Achnowledgements}
{\autor} would like to welcome the following users of the German Raspberry Pi Forum\\ 
(\url{https://forum-raspberrypi.de}):

\textbf{@hyle} \url{https://forum-raspberrypi.de/user/36638-hyle/}:\\
This user gave many valuable hints to the realisation of {\Bezeichnung},
especially the idea to lock the eject push button of the {\CDROM} drive.

\textbf{@smutbert} \url{https://forum-raspberrypi.de/user/21740-smutbert/}:\\
A real expert about ALSA and its installation and configuration.

\textbf{@rpi444} \url{https://forum-raspberrypi.de/user/8097-rpi444/},\\
\textbf{@Tell} \url{https://forum-raspberrypi.de/user/9272-tell/}:\\
Both users showed the right way to tackle compiling large foreign 
C/C++ projects successfully.



\newpage
\section{Conventions of this Manual}
The following design conventions are used for this manual:

\begin{table}[!h]
\centering
\renewcommand{\arraystretch}{2}
\begin{tabular}{|p{3.5cm}|p{10.5cm}|}
\hline
\parbox[c]{3.5cm}{\begin{bclogo}[arrondi = 0.2, logo = \bcdanger, ombre = true, epOmbre = 0.25, couleurOmbre = red!75,blur]{Danger!} 
Text
\end{bclogo}} & Disregarding this warning can result in \textbf{personal injury} and equipment damage!\\
\hline

\parbox[c]{3.5cm}{\begin{bclogo}[arrondi = 0.2, logo = \bcinfo, ombre = true, epOmbre = 0.25, couleurOmbre = black!30,blur]{Caution} 
Text
\end{bclogo}} & Disregarding this warning can result in equipment damage!\\
\hline

\parbox[c]{1em}{\begin{bclogo}[logo = \bclampe, noborder = true]{Hint}
Text
\end{bclogo}} & A note with additional information or relevant explanation of a certain functionality\\
\hline

\button{Button}      & Marking of software buttons\\
\hline
\menuitem{Menu item} & Marking of software menu items\\
\hline
\cmdPi{RPi command}  & command line on the {\RPi}\\
\hline
\cmdPC{PC command}  & command line on a Linux PC\\
\hline
\cmdWin{PC command} & command line on a Windows PC (alternatively to Linux)\\
\hline
\comment{Comment} & comment text in a command line\\
\hline
\stdout{Message} & Marking of software messages\\
\hline
\end{tabular}
\vspace{0.5cm}
\caption{Conventions of this documentation}
\end{table}


%\parbox[c]{1em}{\begin{bclogo}[logo = \bclampe, noborder = true]{Hinweis} Text \end{bclogo}
%\parbox[c]{3.5cm}{\begin{bclogo}[arrondi = 0.2, logo = \bcattention, ombre = true, epOmbre = 0.25, couleurOmbre = black!30,blur]{logo} attention \end{bclogo}}	
%\parbox[c]{3.5cm}{\begin{bclogo}[arrondi = 0.2, logo = \bcbombe, ombre = true, epOmbre = 0.25, couleurOmbre = black!30,blur]{logo} bombe \end{bclogo}}	
%\parbox[c]{3.5cm}{\begin{bclogo}[arrondi = 0.2, logo = \bcbook, ombre = true, epOmbre = 0.25, couleurOmbre = black!30,blur]{logo} book \end{bclogo}}	
%\parbox[c]{3.5cm}{\begin{bclogo}[arrondi = 0.2, logo = \bccalendrier, ombre = true, epOmbre = 0.25, couleurOmbre = black!30,blur]{logo} calendrier \end{bclogo}}	
%\parbox[c]{3.5cm}{\begin{bclogo}[arrondi = 0.2, logo = \bcclefa, ombre = true, epOmbre = 0.25, couleurOmbre = black!30,blur]{logo} clefa \end{bclogo}}	
%\parbox[c]{3.5cm}{\begin{bclogo}[arrondi = 0.2, logo = \bccle, ombre = true, epOmbre = 0.25, couleurOmbre = black!30,blur]{logo} cle \end{bclogo}}	
%\parbox[c]{3.5cm}{\begin{bclogo}[arrondi = 0.2, logo = \bcclesol, ombre = true, epOmbre = 0.25, couleurOmbre = black!30,blur]{logo} clesol \end{bclogo}}	
%\parbox[c]{3.5cm}{\begin{bclogo}[arrondi = 0.2, logo = \bccoeur, ombre = true, epOmbre = 0.25, couleurOmbre = black!30,blur]{logo} coeur \end{bclogo}}	
%\parbox[c]{3.5cm}{\begin{bclogo}[arrondi = 0.2, logo = \bccrayon, ombre = true, epOmbre = 0.25, couleurOmbre = black!30,blur]{logo} crayon \end{bclogo}}	
%\parbox[c]{3.5cm}{\begin{bclogo}[arrondi = 0.2, logo = \bccube, ombre = true, epOmbre = 0.25, couleurOmbre = black!30,blur]{logo} cube \end{bclogo}}	
%\parbox[c]{3.5cm}{\begin{bclogo}[arrondi = 0.2, logo = \bcdanger, ombre = true, epOmbre = 0.25, couleurOmbre = black!30,blur]{logo} danger \end{bclogo}}	
%\parbox[c]{3.5cm}{\begin{bclogo}[arrondi = 0.2, logo = \bcdallemagne, ombre = true, epOmbre = 0.25, couleurOmbre = black!30,blur]{logo} dallemagne \end{bclogo}}	
%\parbox[c]{3.5cm}{\begin{bclogo}[arrondi = 0.2, logo = \bcdautriche, ombre = true, epOmbre = 0.25, couleurOmbre = black!30,blur]{logo} dautriche \end{bclogo}}	
%\parbox[c]{3.5cm}{\begin{bclogo}[arrondi = 0.2, logo = \bcdbelgique, ombre = true, epOmbre = 0.25, couleurOmbre = black!30,blur]{logo} dbelgique \end{bclogo}}	
%\parbox[c]{3.5cm}{\begin{bclogo}[arrondi = 0.2, logo = \bcdbulgarie, ombre = true, epOmbre = 0.25, couleurOmbre = black!30,blur]{logo} dbulgarie \end{bclogo}}	
%\parbox[c]{3.5cm}{\begin{bclogo}[arrondi = 0.2, logo = \bcdfrance, ombre = true, epOmbre = 0.25, couleurOmbre = black!30,blur]{logo} dfrance \end{bclogo}}	
%\parbox[c]{3.5cm}{\begin{bclogo}[arrondi = 0.2, logo = \bcditalie, ombre = true, epOmbre = 0.25, couleurOmbre = black!30,blur]{logo} ditalie \end{bclogo}}	
%\parbox[c]{3.5cm}{\begin{bclogo}[arrondi = 0.2, logo = \bcdluxembourg, ombre = true, epOmbre = 0.25, couleurOmbre = black!30,blur]{logo} dluxembourg \end{bclogo}}	
%\parbox[c]{3.5cm}{\begin{bclogo}[arrondi = 0.2, logo = \bcdpaysbas, ombre = true, epOmbre = 0.25, couleurOmbre = black!30,blur]{logo} dpaysbas \end{bclogo}}	
%\parbox[c]{3.5cm}{\begin{bclogo}[arrondi = 0.2, logo = \bcdz, ombre = true, epOmbre = 0.25, couleurOmbre = black!30,blur]{logo} dz \end{bclogo}}	
%\parbox[c]{3.5cm}{\begin{bclogo}[arrondi = 0.2, logo = \bceclaircie, ombre = true, epOmbre = 0.25, couleurOmbre = black!30,blur]{logo} eclaircie \end{bclogo}}	
%\parbox[c]{3.5cm}{\begin{bclogo}[arrondi = 0.2, logo = \bcetoile, ombre = true, epOmbre = 0.25, couleurOmbre = black!30,blur]{logo} etoile \end{bclogo}}	
%\parbox[c]{3.5cm}{\begin{bclogo}[arrondi = 0.2, logo = \bcfemme, ombre = true, epOmbre = 0.25, couleurOmbre = black!30,blur]{logo} femme \end{bclogo}}	
%\parbox[c]{3.5cm}{\begin{bclogo}[arrondi = 0.2, logo = \bcfeujaune, ombre = true, epOmbre = 0.25, couleurOmbre = black!30,blur]{logo} feujaune \end{bclogo}}	
%\parbox[c]{3.5cm}{\begin{bclogo}[arrondi = 0.2, logo = \bcfeurouge, ombre = true, epOmbre = 0.25, couleurOmbre = black!30,blur]{logo} feurouge \end{bclogo}}	
%\parbox[c]{3.5cm}{\begin{bclogo}[arrondi = 0.2, logo = \bcfeutricolore, ombre = true, epOmbre = 0.25, couleurOmbre = black!30,blur]{logo} feutricolore \end{bclogo}}	
%\parbox[c]{3.5cm}{\begin{bclogo}[arrondi = 0.2, logo = \bcfeuvert, ombre = true, epOmbre = 0.25, couleurOmbre = black!30,blur]{logo} feuvert \end{bclogo}}	
%\parbox[c]{3.5cm}{\begin{bclogo}[arrondi = 0.2, logo = \bcfleur, ombre = true, epOmbre = 0.25, couleurOmbre = black!30,blur]{logo} fleur \end{bclogo}}	
%\parbox[c]{3.5cm}{\begin{bclogo}[arrondi = 0.2, logo = \bchomme, ombre = true, epOmbre = 0.25, couleurOmbre = black!30,blur]{logo} homme \end{bclogo}}	
%\parbox[c]{3.5cm}{\begin{bclogo}[arrondi = 0.2, logo = \bchorloge, ombre = true, epOmbre = 0.25, couleurOmbre = black!30,blur]{logo} horloge \end{bclogo}}	
%\parbox[c]{3.5cm}{\begin{bclogo}[arrondi = 0.2, logo = \bcicosaedre, ombre = true, epOmbre = 0.25, couleurOmbre = black!30,blur]{logo} icosaedre \end{bclogo}}	
%\parbox[c]{3.5cm}{\begin{bclogo}[arrondi = 0.2, logo = \bcinfo, ombre = true, epOmbre = 0.25, couleurOmbre = black!30,blur]{logo} info \end{bclogo}}	
%\parbox[c]{3.5cm}{\begin{bclogo}[arrondi = 0.2, logo = \bcinterdit, ombre = true, epOmbre = 0.25, couleurOmbre = black!30,blur]{logo} interdit \end{bclogo}}	
%\parbox[c]{3.5cm}{\begin{bclogo}[arrondi = 0.2, logo = \bclampe, ombre = true, epOmbre = 0.25, couleurOmbre = black!30,blur]{logo} lampe \end{bclogo}}	
%\parbox[c]{3.5cm}{\begin{bclogo}[arrondi = 0.2, logo = \bcloupe, ombre = true, epOmbre = 0.25, couleurOmbre = black!30,blur]{logo} loupe \end{bclogo}}	
%\parbox[c]{3.5cm}{\begin{bclogo}[arrondi = 0.2, logo = \bcneige, ombre = true, epOmbre = 0.25, couleurOmbre = black!30,blur]{logo} neige \end{bclogo}}	
%\parbox[c]{3.5cm}{\begin{bclogo}[arrondi = 0.2, logo = \bcnote, ombre = true, epOmbre = 0.25, couleurOmbre = black!30,blur]{logo} note \end{bclogo}}	
%\parbox[c]{3.5cm}{\begin{bclogo}[arrondi = 0.2, logo = \bcnucleaire, ombre = true, epOmbre = 0.25, couleurOmbre = black!30,blur]{logo} nucleaire \end{bclogo}}	
%\parbox[c]{3.5cm}{\begin{bclogo}[arrondi = 0.2, logo = \bcoctaedre, ombre = true, epOmbre = 0.25, couleurOmbre = black!30,blur]{logo} octaedre \end{bclogo}}	
%\parbox[c]{3.5cm}{\begin{bclogo}[arrondi = 0.2, logo = \bcoeil, ombre = true, epOmbre = 0.25, couleurOmbre = black!30,blur]{logo} oeil \end{bclogo}}	
%\parbox[c]{3.5cm}{\begin{bclogo}[arrondi = 0.2, logo = \bcorne, ombre = true, epOmbre = 0.25, couleurOmbre = black!30,blur]{logo} orne \end{bclogo}}	
%\parbox[c]{3.5cm}{\begin{bclogo}[arrondi = 0.2, logo = \bcours, ombre = true, epOmbre = 0.25, couleurOmbre = black!30,blur]{logo} ours \end{bclogo}}	
%\parbox[c]{3.5cm}{\begin{bclogo}[arrondi = 0.2, logo = \bcoutil, ombre = true, epOmbre = 0.25, couleurOmbre = black!30,blur]{logo} outil \end{bclogo}}	
%\parbox[c]{3.5cm}{\begin{bclogo}[arrondi = 0.2, logo = \bcpeaceandlove, ombre = true, epOmbre = 0.25, couleurOmbre = black!30,blur]{logo} peaceandlove \end{bclogo}}	
%\parbox[c]{3.5cm}{\begin{bclogo}[arrondi = 0.2, logo = \bcpluie, ombre = true, epOmbre = 0.25, couleurOmbre = black!30,blur]{logo} pluie \end{bclogo}}	
%\parbox[c]{3.5cm}{\begin{bclogo}[arrondi = 0.2, logo = \bcplume, ombre = true, epOmbre = 0.25, couleurOmbre = black!30,blur]{logo} plume \end{bclogo}}	
%\parbox[c]{3.5cm}{\begin{bclogo}[arrondi = 0.2, logo = \bcpoisson, ombre = true, epOmbre = 0.25, couleurOmbre = black!30,blur]{logo} poisson \end{bclogo}}	
%\parbox[c]{3.5cm}{\begin{bclogo}[arrondi = 0.2, logo = \bcrecyclage, ombre = true, epOmbre = 0.25, couleurOmbre = black!30,blur]{logo} recyclage \end{bclogo}}	
%\parbox[c]{3.5cm}{\begin{bclogo}[arrondi = 0.2, logo = \bcrosevents, ombre = true, epOmbre = 0.25, couleurOmbre = black!30,blur]{logo} rosevents \end{bclogo}}	
%\parbox[c]{3.5cm}{\begin{bclogo}[arrondi = 0.2, logo = \bcsmiley-bonnehumeur, ombre = true, epOmbre = 0.25, couleurOmbre = black!30,blur]{logo} smiley-bonnehumeur \end{bclogo}}	
%\parbox[c]{3.5cm}{\begin{bclogo}[arrondi = 0.2, logo = \bcsmiley-mauvaisehumeur, ombre = true, epOmbre = 0.25, couleurOmbre = black!30,blur]{logo} smiley-mauvaisehumeur \end{bclogo}}	
%\parbox[c]{3.5cm}{\begin{bclogo}[arrondi = 0.2, logo = \bcsoleil, ombre = true, epOmbre = 0.25, couleurOmbre = black!30,blur]{logo} soleil \end{bclogo}}	
%\parbox[c]{3.5cm}{\begin{bclogo}[arrondi = 0.2, logo = \bcstop, ombre = true, epOmbre = 0.25, couleurOmbre = black!30,blur]{logo} stop \end{bclogo}}	
%\parbox[c]{3.5cm}{\begin{bclogo}[arrondi = 0.2, logo = \bctakecare, ombre = true, epOmbre = 0.25, couleurOmbre = black!30,blur]{logo} takecare \end{bclogo}}	
%\parbox[c]{3.5cm}{\begin{bclogo}[arrondi = 0.2, logo = \bctetraedre, ombre = true, epOmbre = 0.25, couleurOmbre = black!30,blur]{logo} tetraedre \end{bclogo}}	
%\parbox[c]{3.5cm}{\begin{bclogo}[arrondi = 0.2, logo = \bctrefle, ombre = true, epOmbre = 0.25, couleurOmbre = black!30,blur]{logo} trefle \end{bclogo}}	
%\parbox[c]{3.5cm}{\begin{bclogo}[arrondi = 0.2, logo = \bctrombone, ombre = true, epOmbre = 0.25, couleurOmbre = black!30,blur]{logo} trombone \end{bclogo}}	
%\parbox[c]{3.5cm}{\begin{bclogo}[arrondi = 0.2, logo = \bcvaletcoeur, ombre = true, epOmbre = 0.25, couleurOmbre = black!30,blur]{logo} valetcoeur \end{bclogo}}	
%\parbox[c]{3.5cm}{\begin{bclogo}[arrondi = 0.2, logo = \bcvelo, ombre = true, epOmbre = 0.25, couleurOmbre = black!30,blur]{logo} velo \end{bclogo}}	
%\parbox[c]{3.5cm}{\begin{bclogo}[arrondi = 0.2, logo = \bcyin, ombre = true, epOmbre = 0.25, couleurOmbre = black!30,blur]{logo} yin \end{bclogo}}	
%

%% Diese Auflistung k�nnte irgendwann notwendig werden,
%% wenn auf Github verschiedene Versionsst�nde herumgeistern und die
%% einzelnen Nutzer evtl. verschiedene St�nde vorliegen haben...
%Ausgabest�nde dieser Dokumentation:
%
%\begin{tabular}{llp{0.70\textwidth}}
%\textbf{Version 1} & 20.02.2016 & Erstversion f�r Programmversion V0.1\\
%\textbf{Version 2} & 01.07.2016 & zus�tzliche Hinweise zur Software-Installation von Version V0.1\\
%\textbf{Version 3} & 18.08.2017 & {\Bezeichnung} {\Version}\\
%\end{tabular}


\newpage
\section{Abstract} % Brief Description
{\Bezeichnung} is a CD player with built-in stereo amplifier
(\hardware{HifiBerry MiniAMP}) and loudspeakers. It has got some 
similarity (at least in technical terms) to the devices called
\textit{boom boxes} and especially in Germany also called 
\textit{Ghettoblaster} and which are known since the 1980s. The
{\Bezeichnung} consists of a {\RPi} 3B which is internally connected to 
a {\CDROM} drive (\hardware{LG \LGdrive}, which is actually a DVD 
writer). The {\Bezeichnung} is operated via the 7-inch {\raspidisplay} 
issued by the {\foundation}. At the back side there is a connector for 
external loudspeakers. Two of the {\RPi}'s USB connectors and its 
Ethernet socket are accessible by the user via housing connectors.\\
The power supply is done by a mains power cord. Inside the 
{\Bezeichnung} there is a built-in power supply device (\hardware{
Meanwell RS-25-05}: 5V, 5A) which provides enough power reserves.

{\Bezeichnung} uses the open source software {\audacious} in 
{\audaciousStable} for playback of audio CDs.

\subsection*{Prospect on Further Enhancements}
\begin{compactitem}
\item{battery operation}
\item{Push buttons and rotary encoder as hardware via GPIO for volume control etc.}
\item{media player for USB flash drives}
\item{DVD playback (using kodi)}
\item{Bluetooth receiver}
\item{FM and/or DAB+ radio}
\end{compactitem}

\newpage
\section{Parts List} % Bill of Material
\begin{table}[!h]
\centering
\renewcommand{\arraystretch}{1.25}
\begin{tabular}{|r|l|p{2.6cm}|p{4.5cm}|l|p{2.7cm}|}
\hline
\textbf{x}	&	\textbf{Manufacturer}	&	\textbf{Type}	&	\textbf{Description}				&	\textbf{Distributor}	&	\textbf{Order\#}\\
\hline
		1		&	bopla				&	68626120		&	Botego BO 62612\newline
%																Geh�use geschlossen\newline
																308mm x 257mm x 81mm				&	?						&	--\\
		1		&	Schurter			&	6762			&	Power Supply Combination			&	B�rklin					&	41 F 139\\
		2		&	RND					&	170-00020		&	Microfuse ``time delay'' 1A			&	reichelt				&	RND 170-00020\\
		1		&	MeanWell			&	RS 25-5			&	Switching Power Supply 25W, 5V, 5A	&	reichelt				&	SNT RS 25 5\\
		1		&	Foundation			&	-				&	{\raspidisplay}						&	raspiprojekt			&	TS7DSI\\
		1		&	Foundation			&	RPi 3B			&	Raspberry Pi 3B						&	raspiprojekt			&	RASPI3B\\
		1		&	HifiBerry			&	MiniAMP	V1.0	&	Stereo Amplifier\newline
																2x3W maximum power					&	reichelt				&	RPI HB\newline MINI AMP\\
		2		&	Visaton				&	SC 8 N 8Ohm		&	Speaker 30W, 8Ohm					&	reichelt				&	VIS SC 8N-8\\
		2		&	Visaton				& 	GRILLE FRS 8	&	Speaker Cover\newline
																82mm x 82mm							&	RS 						&	4538953\\
		1		&	?					&					&	Speaker Connector					&	RS 					 	&	392683\\
		1       &	Marquardt			&	1839.0105		&	Rocker Switch 2P I/O/II\newline
																30mm x 22mm black					&	RS 					 	&	7410823\\

		1		&	LG					&	\LGdrive		&	Slim Line DVD-Writer				&	\textit{local store}	&	\\
		2		&	Neutrik				&	NAUSB-W-B		&	USB Socket Type ``A''				&	reichelt				&	NAUSB-WB\\
		1		&	Neutrik				&	NE8FDX-P6		&	Ethernet (LAN) Socket Cat.6A		& 	reichelt				&	CAT6A BU BK\\
		1		&	schlizb�da			&	--				&	Relay Control ``eject-lock'' 		&							&	\\
\hline
\end{tabular}
\vspace{0.5cm}
\caption{Parts list}
\end{table}

Sundries like screws, connection cables (ethernet, USB, \dots) aren't 
listed in this parts list. 


\section{Technical Specifications}
\renewcommand{\arraystretch}{1.25}

\subsection{Power Supply}
\begin{tabular}{p{5.0cm}p{10.0cm}}
Voltage Range				&	88VAC -- 264VAC\\
Frequency Range				&	47Hz -- 63Hz\\
AC Current					&	0.7A/115VAC 0,4A/230VAC\\
DC Voltage					&	5V\\
Rated Current				&	5A\\
Rated Power					&	25W\\
\end{tabular}

\subsection{Audio Amplifier (HifiBerry MiniAMP)}
\begin{tabular}{p{5.0cm}p{10.0cm}}
Amplifier Class				&	HifiBerry MiniAMP V1.0 (class-D amplifier)\\
Music Power					&	2 x 3W (max.)\\
Sample Rate					&	44,1kHz -- 192kHz\\
\end{tabular}

\subsection{\RPi}
\begin{tabular}{p{5.0cm}p{10.0cm}}
{\RPi} Version				&	{\RPi} 3B\\
SoC (Broadcom)				&	BCM2837\\
Architecture				&	ARM Cortex-A53 (quad core)\\
Clock Rate CPU				&	1200MHz\\
Clock Rate GPU				&	300MHz/400MHz\\
Main Memory					&	1GB\\
Non-volatile Memory			&	depends on the used micro SD card\newline
								To run ``Raspbian Stretch Desktop'' an 8GB sized SD card is necessary as minimal storage capacity\\
\end{tabular}
\textbf{Used GPIOs}\\
\begin{tabular}{p{5.0cm}p{10.0cm}}
GPIO 4						&	Pin \textbf{7}: signal for relay control of eject-lock\\
GPIO 2, 3					&   Pins \textbf{3}, \textbf{5}: MiniAMP I2C bus\\
GPIO 18 -- 21				&   Pins \textbf{12}, \textbf{35}, \textbf{38}, \textbf{40}: MiniAMP I2S bus\\
GPIO 26						&	Pin \textbf{37}: MiniAmp shut down power stage\\
ID SDA, ID SCL				&	Pins \textbf{27}, \textbf{28}: I2C EEPROM containing device type data
\end{tabular}

\subsection{CD Drive (LG \LGdrive)}
\begin{tabular}{p{5.0cm}p{10.0cm}}
Supported Discs	&				\mbox{DVD-ROM} (Single/Dual), \mbox{DVD-RW}, 
								\mbox{DVD-R}, \mbox{DVD+RW}, \mbox{DVD+R}, 
								\mbox{DVD+R} Double layer, \mbox{DVD-R} Dual 
								layer, \mbox{DVD-RAM}, \mbox{M-Disc} 
								\mbox{(DVD+R SL)}, \mbox{CDDA} (CD Digital Audio)
								\& \mbox{CD-Extra}, \mbox{CD-Plus}, 
								\mbox{CD-ROM}, \mbox{CD-ROM XA-Ready}, 
								\mbox{CD-I FMV}, \mbox{CD-TEXT}, 
								\mbox{CD-Bridge}, \mbox{CD-R}, \mbox{CD-RW},
								\mbox{Photo-CD} (Single- \& Multi-Session),
								\mbox{Video CD}, \mbox{DVD-VIDEO}\\
\end{tabular}
\begin{tabular}{p{5.0cm}p{10.0cm}}
Read Speed &					DVD-R/RW/ROM: 8x/8x/8x max.\newline
								DVD-R DL: 8x max.\newline
								DVD-RAM (Ver.2.2 \& Higher): 6x max.\newline
								M-Disc (DVD+R SL): 8x max.\newline
								DVD-Video (CSS Compliant Disc) : 4x max.\newline
								DVD+R/+RW: 8x/8x max.\newline
								DVD+R DL: 8x max.\newline
								CD-R/RW/ROM: 24x/24x/24x max.\newline
								CD-DA (DAE): 24x max.\\
\end{tabular}
\begin{tabular}{p{5.0cm}p{10.0cm}}
Write Speed	&					DVD-R: 2x, 4x, 8x\newline
								DVD-R DL: 2x, 4x, 6x\newline
								DVD-RW: 2x, 4x, 6x\newline
								DVD-RAM (Ver. 2.2 \& higher): 2x, 3x, 5x\newline
								M-Disc (DVD+R SL): 4x\newline
								DVD+R: 2.4x, 4x, 8x\newline
								DVD+R DL: 2.4x, 4x, 6x\newline
								DVD+RW: 2.4x, 3.3x, 4x, 8x\newline
								CD-R: 10x, 16x, 24x\newline
								CD-RW: 4x, 10x, 16x, 24x\\
\end{tabular}
\begin{tabular}{p{5.0cm}p{10.0cm}}
Interface					&	USB 2.0\\
Input Voltage				&	5V DC\\
Power Dissipation			&	1,6A\\
\end{tabular}

\textbf{Data Transfer Rate}\\		
\begin{tabular}{p{5.0cm}p{10.0cm}}
Sustained					&	CD-ROM: 3,600 kB/s (24x max)\newline
								DVD-ROM: 11.08 MB/s (8x max)\newline
								CD-ROM: 140 ms (typisch)\newline
								DVD-ROM: 160 ms (typisch)\newline
								DVD-RAM: 200 ms (typisch)\\
Buffer capacity				&	0.75 MB\\
MTBF						&	60000 Power On Hours (Duty Cycle 10\%)\\
\end{tabular}

\textbf{Operating Environment}\\
\begin{tabular}{p{5.0cm}p{10.0cm}}
Temperature					&	5�C to 40�C\\
Humidity					&	15\% to 85\%\\
\end{tabular}

\textbf{Storage Environment}\\
\begin{tabular}{p{5.0cm}p{10.0cm}}
Temperature					&	-30�C to 60�C\\
Humidity					&	10\% to 90\%\\
\end{tabular}
